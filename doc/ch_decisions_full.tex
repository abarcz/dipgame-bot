Platforma dipGame udostępnia ramy dające dobrą podstawę do implementacji agenta-bota grającego w grę w Dyplomację. Sugerowany szkielet bota opiera się o implementację trzech ewaluatorów - ewaluatora prowincji, ewaluatora rozkazów oraz ewaluatora opcji. Pierwszy z nich ocenia prowincje, drugi zaś rozkazy, które wykonać mogą poszczególne jednostki posiadane przez gracza. Na podstawie ocen tych dwóch ewaluatorów budowane są opcje - czyli kombinacje określonej ilości najlepszych rozkazów wykonywanych przez wszystkie dostępne jednostki. Następnie następuje wybór najlepszej spośród danej liczby najlepszych spośród ocenionych opcji. Za wprowadzanie prowincji, rozkazów oraz konstruowanie i wprowadzanie opcji do ewaluacji odpowiada platforma dipGame. Naszym zadaniem było dokonanie ewaluacji oraz ostateczna ocena i wybór konkretnej opcji - czyli wybór najlepszej kombinacji ruchów jednostek. Ciekawostką może tu być fakt, że platforma proponowała ewaluatorom także rozkazy niemożliwe do przeprowadzenia, co sugeruje, iż gdzieś na poziomie jej implementacji tkwi błąd. Konieczne okazało się w związku z tym odrzucanie niepoprawnych rozkazów na poziomie logiki agenta.

    Zadaniem pierwszego z ewaluatorów, tj. ewaluatora prowincji, jest ocena wartości prowincji. Nasza implementacja rozwija pomysł oceny prowincji - oprócz pojedyńczej oceny wymaganej przez platformę wprowadziliśmy szereg współczynników używanych przez pozostałe ewaluatory do polepszenia jakości oceny. Współczynniki te to wartość ataku i obrony prowincji, a także siły  konkurencji o prowincję. Atak i obrona mówią zasadniczo jak bardzo opłacalny jest atak lub obrona danej prowincji (i przekłada się na ocenę regionu na potrzeby platformy). Oba te parametry wiążą się z ilością punktów zaopatrzeniowych właściciela oraz liczbą przylegających punktów zaopatrzeniowych. Dwa pozostałe parametry wiążą się już z opłacalnością ataku albo obrony prowincji w kontekście przebywających nieopodal jednostek. Siła jest tym większa, im więcej przyjaznych jednostek może wspierać dany region - konkurencja odwrotnie, jest proporcjonalna do ilości jednostek zagrażających prowincji. Jako jednostki przyjazne są brane pod uwagę zarówno jednostki danego gracza jak i jego sojuszników. Jednostki sojuszników nie są traktowane jako wrogie.

	Oprócz oceny taktycznej na poziomie możliwości ataku/obrony pojedynczego gracza, brane są również pod uwagę możliwości ataku w ramach sojuszy. Wspólne akcje są negocjowane z sojusznikami (więcej na ten temat w dalszej części sprawozdania)  i rozkazy w ten sposób wynegocjowane są oceniane wyżej przez ewaluator. Dodatkowo, wyżej oceniane są akcje przeciwko wspólnym wrogom (w obrębie sojuszy), co ma na celu koordynację ataku sprzymierzonych państw na wspólnego wroga, nawet jeśli pojedynczy wspólny atak (na jedną prowincję) nie jest możliwy ze względu na ukształtowanie granic.
