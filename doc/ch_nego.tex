Kluczowym aspektem działania agenta jest komunikacja z innymi agentami, co w przypadku bota do gry w Dyplomację manifestuje się pod postacią negocjacji z innymi graczami. Bot realizowany w ramach naszego projektu realizuje tę funkcjonalność na kilku poziomach, związanych luźno z poziomami języka L, na którym oparta jest komunikacja w obrębie platformy dipGame.

    Pierwszym poziomem języka L jest negocjowanie układów. W szczególności w sferze naszego zainteresowania znalazły się trzy spośród dostępnych ofert: zawieranie sojuszu pomiędzy dwoma graczami przeciwko wspólnemu wrogowi, zawieszanie broni oraz negocjowanie łączonych ataków. Podczas gry agent ocenia na podstawie poziomu zaufania z którym z innych graczy chciałby się sprzymierzyć - jeżeli dany gracz jest “lubiany” przez agenta istnieje szansa na wysłanie do niego oferty sojuszu przeciwko jednemu z graczy stanowiących potencjalne zagrożenie.

	    Po zawarciu sojuszu przede wszystkim z większą intensywnością podejmowane są akcje przeciwko wrogom. Jest to pierwsza płaszczyzna współracy między dwoma agentami - gracz uznany za wspólnego wroga jest częściej atakowany przez wszystkie wspólnie działające agenty, tym samym zwiększając szansę na sukcesy podczas wojny. Jest to podstawowe, ale skuteczne podejście.

		    Rozwinięciem pomysłu wspólnego prowadzenia wojny jest dodanie możliwości łączenia ataków. Nasza implementacja wykorzystuje pierwszą warstwę języka L do negocjowania z sojusznikami wsparcia dla przeprowadzanych przez agenta ataków. Przed rozpoczęciem ewaluacji poszczególnyh akcji przeprowadzana jest wstępna ewaluacja prowincji w której wartości siły i konkurencji prowincji rozpatrywane są nie tylko w kontekście własnych jednostek i już wynegocjowanych ataków, ale także wszystkich sojuszniczych jednostek. W ten sposób typowane są obszary najbardziej nadające się do wspólnego ataku - jeżeli wartość prowincji jest wysoka, zaś ocena wyliczonych parametrów sugeruje, że jej podbój możliwy byłby tylko z pomocą sojusznika, to proponowany jest wspólny atak. Jeżeli propozycja zostanie rozpatrzona pomyślnie, to oba agenty zwiększą ocenę przydzielaną rozkazowi ataku i wsparcia ataku dla wynegocjowanej, wspólnej akcji. W ten sposób znacząco zwiększana jest szansa na przeprowadzanie wspólnego ataku. Jednocześnie jeżeli w toku ewaluacji okaże się, że dla danego agenta wynegocjowana akcja jest w dalszym ciągu znacząco gorsza od innej akcji (co znamionuje większa wartość danej akcji wobec zwiększonej wartości oceny akcji związanej ze wspólnym atakiem), to zachowana jest możliwość rezygnacji ze wspólnej akcji - ostatecznie każda z potęg w Dyplomacji dba przede wszystkim o siebie!

			    Dalsze dwa poziomy języka L - drugi i trzeci - związane są odpowiednio z dzieleniem się informacjami i pytaniem o informacje bezpośrednie. Te poziomy wykorzystaliśmy do rozbudowy wiedzy agenta o sytuacji w grze. Agent odpytuje innych graczy o ich sojusze i na tej podstawie buduje swoją wiedzę o powiązaniach pomiędzy innymi graczami. W szczególności daje to szansę na budowę stronnictw działających wspólnie agentów - w myśl zasady “wróg mojego wroga jest moim przyjacielem”.

				    Pierwotnie zakładaliśmy implementację agenta z wykorzystaniem poziomów języka L do czwartego włącznie. Poziom czwarty pozwala na wymianę informacji niebezpośrednich, np. zapytań o to, czy dany gracz przekazał daną informację innemu graczowi. Niestety okazało się, że poziom ten nie został zaimplementowany w ramach platformy dipGame - obsługuje ona tylko trzy pierwsze poziomy języka L. W związku z tym obsługa wszystkich założonych poziomów w ramach naszego agenta okazała się niemożliwa.

					Najpoważniejszy problem implementacyjny zaczął pojawiać się wraz z rozszerzaniem możliwości negocjacyjnych agenta, a więc wraz ze zwiększającą się ilością przesyłanych komunikatów. O ile agent korzystający ze znikomej ilości komunikacji działał bez problemu nawet w kilku instancjach na raz, to zwiększanie ilości komunikatów spowodowało zawieszanie się agenta (ustępujące po kilkudziesięciu sekundach bądź wcale), tym częstsze im więcej komunikatów było przesyłanych. Próby odkrycia przyczyny ujawniły, że agent zawiesza się na próbie odczytu z socketa odpowiedzialnego za komunikację z serwerem gry. Może to wskazywać na istnienie błędu w platformie dipGame - na poziomie biblioteki implementacji agenta bądź w samym serwerze gry.
