Agent prowadzący negocjacje z innymi graczami powinien zachowywać się racjonalnie wobec ich postępowania oraz własnej oceny zagrożenia z ich strony. W tym celu został zaimplementowany model relacji agenta z pozostałymi graczami oraz jego baza wiedzy, opisująca całość jego wiedzy i przekonań o świecie gry. Model ten pozwala na ocenę pozostałych graczy przy użyciu skali punktowych, wyrażających niezależnie:
\begin{itemize}
	\item{zaufanie do drugiego gracza}
	\item{ocenę siły drugiego gracza}
\end{itemize}

\section{Zaufanie}
Zaufanie do drugiego gracza odzwierciedla jego przeszłe zachowania wobec analizowanego agenta. Gracz, z którym agent wcześniej nie wchodził w żadne interakcje jest dla agenta neutralny. Wszelkie odmowy porozumienia, łamanie porozumień, wrogie zachowania (np. zaatakowanie sprzymierzonego gracza) skutkują obniżeniem zaufania. Wszelkie zawarte z danym graczem uzgodnienia powodują zwiększenie zaufania do niego. Wysoki poziom zaufania do innego gracza skutkuje gotowością do zawierania z nim kolejnych porozumień. Niski poziom zaufania skutkuje zwiększoną gotowością do zaatakowania danego gracza oraz zawierania sojuszy przeciwko niemu. Gracze uznawani za neutralnych są obojętni, w szczególności sojusze przeciwko nim są mało interesujące, natomiast mogą zostać nowymi sojusznikami w przypadku posiadania wspólnego wroga (państwa przeciwko któremu obaj gracze toczą wojnę).

\section{Ocena siły}
W celu oceny zagrożenia ze strony drugiego gracza została zaimplementowana skala siły, oparta na podliczeniu prowincji danego gracza, jego jednostek a także prowincji macierzystych, w których może on budować nowe jednostki. Ocena siły gracza brana jest pod uwagę podczas zawierania sojuszy (im silniejszy gracz tym bardziej interesujący sojusz) a także w momencie ataku (atak na silnego gracza wiąże się z większym ryzykiem).

Zaimplementowana baza wiedzy pozwala zachowywać informację o zdarzeniach ze świata gry, np. wynikach negocjacji, atakach oraz informacje uzyskane od innych graczy, np. dotyczące sojuszy między innymi państwami.
