Dla ułatwienia obserwacji wiedzy, przekonań i poczynań agenta został zaimplementowany graficzny interfejs użytkownika. Ze względu na nieczytelność informacji wypisywanych w głównym logu serwera gry, dodatkowo GUI oferuje podgląd zdarzeń świata gry zapisanych w czytelny dla człowieka sposób, przy czym akcje podejmowane przez obserwowanego gracza opatrzone są komentarzem, którym może być np powód odrzucenia propozycji sojuszu. GUI odzwierciedla przekonania i wiedzę pojedynczego (obserwowanego) agenta na temat świata gry.

W szczególności GUI pokazuje dla każdego gracza biorącego udział w grze:
\begin{itemize}
	\item{listę jego prowincji}
	\item{listę jego jednostek}
	\item{jego sojusze i toczone wojny (na tyle na ile obserwowany agent posiada wiedzę na temat danego agenta)}
\end{itemize}

Dla obserwowanego agenta, GUI pokazuje dodatkowo:
\begin{itemize}
	\item{jego zaufanie do poszczególnych graczy, wyrażone w punktach}
	\item{ocenę siły pozostałych graczy z jego punktu widzenia}
\end{itemize}

Dodatkowo, GUI pozwala na:
\begin{itemize}
	\item{pracę krokową - pauza po każdej turze gry lub tryb automatyczny}
	\item{podgląd w.w. przejrzystego logu zdarzeń}
\end{itemize}
